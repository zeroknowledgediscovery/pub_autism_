\DeclarePairedDelimiter{\ceil}{\lceil}{\rceil}
\renewcommand{\ceil}[1]{\left\lceil{#1}\right\rceil}

\section{Probabilsitic Finite State Automata Inference}
\label{sec:PFSA}

\subsection{Probabilistic Finite-State Automaton}
\label{subsec:DEFN_PFSA}
Let $\Sigma$ be a finite alphabet of symbols with size $\abs{\Sigma}$. The set of sequences of length $d$ over $\Sigma$ is denoted by $\Sigma^d$. The set of finite but unbounded sequences over $\Sigma$ is denoted by $\Sigma^\star$, the Kleene star operation \cite{hopcroft2008introduction}, \ie~$\Sigma^{\star} = \bigcup_{d=0}^{\infty}\Sigma^{d}$. We use lower case Greek, for example $\sigma$ or $\tau$, for symbols in $\Sigma$, and lower case Latin, for example $x$ or $y$, for sequences of symbols, \ie~$x=\sigma_1\sigma_2\dots\sigma_n$. We use $\abs{x}$ to denote the length of $x$.  The empty sequence is denoted by $\lambda$. 

We denote the set of strictly infinite sequences over $\Sigma$ by $\Sigma^\omega$, and the set of strictly infinite sequences having $x$ as prefix by $x\Sigma^\omega$. Let $\SR = \set{x\Sigma^\omega: x \in \Sigma^\star}\cup\set{\emptyset}$, we can verify that $\SR$ is a semiring \cite{klenke2013probability} over $\Sigma^{\omega}$. We use $\mathcal{F}$ to denote the sigma algebra generated by $\mathcal{S}$.

\begin{defn}[Stochastic Process over $\Sigma$]
\label{defn:StochasticProcessOverSigma}
A stochastic process over a finite alphabet $\Sigma$ is a collection of $\Sigma$-valued random variables $\set{X_t}_{t\in\mathbb{N}}$ indexed by positive integers~\cite{doob1990stochastic}.
\end{defn}
We are specifically interested in processes in which the $X_i$s are not necessarily independently distributed.

\begin{defn}[Sequence-Induced Measure and Derivative]
\label{defn:MeasureAndDeriv}
For a process $\mathscr{P}$, let $Pr_{\mathcal{P}}(x)$ or simply $Pr(x)$ denote the probability $\mathscr{P}$ producing a sample path prefixed by $x$. The \textbf{measure $\mu_x$ induced by a sequence $x\in \Sigma^{\star}$} is the extension \cite{klenke2013probability} to $\mathcal{F}$ of the premeasure defined on the semiring $\SR$ given by
\cgather{
%\label{eq:DefinitionOfPremeasure}
 \forall x,y\in\Sigma^{\star},   \mu_x\paren{y\Sigma^{\omega}} \triangleq \frac{Pr\paren{xy}}{Pr(x)}, \textrm{ if } Pr(x) > 0
}For any $d\in\mathbb{N}$, the \textbf{$d$-th order derivative} of a sequence $x$, written as $\phi^{d}_{x}$, is defined to be the marginal distribution of $\mu_{x}$ on $\Sigma^d$, with the entry indexed by $y$ denoted by $\phi^{d}_{x}(y)$. The first-order derivative is called the \textbf{symbolic derivative} and is denoted by $\phi_{x}$ for short.
\end{defn}

\begin{defn}[Probabilistic Nerode Equivalence and Causal States \cite{chattopadhyay2008structural}]
\label{defn:NerodeEquiv}
For any pair of sequences $x, y\in\Sigma^{\star}$, $x$ is equivalent to $y$, written as $x\sim y$, if and only if either $Pr(x) = Pr(y) = 0$, or $\mu_x = \mu_y$. The equivalence class of a sequence $x$ is denoted by $[x]$ and is called a \textbf{causal state} \cite{chattopadhyay2014data}. The cardinality of the set of causal states is called the \textbf{probabilistic Nerode index}, or the Nerode index for simplicity.
\end{defn}
We can see from the definition that causal states captures how the history of a process influences its future. Since the probabilistic Nerode equivalence is right invariant, it gives rise naturally to a automaton structure introduced below.

\begin{defn}[Probabilistic Finite-State Automaton (PFSA)]
\label{defn:PFSA}
A PFSA $G$ is defined by a quadruple $\paren{Q, \Sigma, \delta, \pitilde}$, where $Q$ is a finite set, $\Sigma$ is a finite alphabet, $\delta: Q\times\Sigma \rightarrow \Sigma$ is called the transition map, and $\pitilde: Q \rightarrow \mathbf{P}_\Sigma$, where $\mathbf{P}_\Sigma$ is the space of probability distributions over $\Sigma$, is called the transition probability. The entry of $\pitilde(q)$ indexed by $\sigma$ is denoted by $\pitilde(q,\sigma)$. 
\end{defn}
\begin{defn}[Transition and Observation Matrices]
The transition matrix $\Pi$ is the $|Q|\times|Q|$ matrix with the entry indexed by $q, q'$, written as $\pi_{q, q'}$, satisfying
\cgather{
    \pi_{q, q'} \triangleq \sum_{\mathclap{\set{\sigma\in\Sigma| \delta(q, \sigma) = q'}}}\pitilde(q, \sigma)
}and the observation matrix $\Pitilde$ is a $|Q|\times|\Sigma|$ matrix with the entry indexed by $q, \sigma$ equaling $\pitilde(q, \sigma)$.
\end{defn}
We note that both $\Pi$ and $\Pitilde$ are stochastic, \ie~non-negative with rows summing up to $1$. 
%
\begin{defn}[Extension of $\delta$ and $\pitilde$ to $\Sigma^{\star}$]
For any $x = \sigma_1\dots\sigma_k$, $\delta(q, x)$ is defined recursively by 
\cgather{
    \delta(q, x) \triangleq \delta\paren{\delta\paren{q, \sigma_1\dots\sigma_{k-1}}, \sigma_k}
}with $\delta(q, \lambda) = q$, and $\pitilde(q, x)$ is defined recursively by
\cgather{
    \pitilde(q, x) \triangleq \prod_{i=1}^{k}\pitilde\paren{\delta\paren{q, \sigma_1\dots\sigma_{i-1}}, \sigma_i}
}with $\pitilde(q, \lambda) = 1$. 
\end{defn}
%
\begin{defn}[Strongly Connected PFSA]
\label{defn:StrongConn}
We say a PFSA is strongly connected if the underlying directed graph is strongly connected \cite{bondy2008graph}. More precisely, a PFSA $G = \paren{Q, \Sigma, \delta, \pitilde}$ is strongly connected if for any pair of distinct states $q$ and $q'\in{Q}$, there is an $x\in\Sigma^{\star}$ such that $\delta(q, x) = q'$.
\end{defn}
We assume all PFSA in the discussions in the sequel are strongly connected if not specified otherwise. For strongly connected PFSA $G$, there is a unique probability distribution over $Q$ that satisfies $\mathbf{v}^{T}\Pi = \mathbf{v}^{T}$. This is  the \textbf{stationary distribution} \cite{vidyasagar2014hidden,kai1967markov_StDis} of $G$ and is denoted as $\wp_G$, or $\wp$ if $G$ is understood. 

\begin{defn}[$\Gamma$-Expression]
\label{defn:GammaExpr}
We can encode the information contained in $\delta$ and $\pitilde$ by a set of $|Q| \times |Q|$ matrices $\boldsymbol{\Gamma} = \set{\Gamma_\sigma| \sigma\in\Sigma}$, where
\cgather{
    \Gamma_\sigma\big|_{q, q'} \triangleq \left\{
        \begin{array}{ll}
        \pitilde(q, \sigma) & \textrm{if } \delta(q, \sigma) = q',\\
        0 & \textrm{if otherwise}.
        \end{array}\right.
}$\Gamma_\sigma$ is called \textbf{event-specific transition matrix}, with the event being that $\sigma$ is current the output. $\Gamma_\sigma$ can also be extended to arbitrary $x\in\Sigma^{\star}$ by defining $\Gamma_{x} = \prod_{i = 1}^{k}\Gamma_{\sigma_i}$ with $\Gamma_{\lambda} = I$.
\end{defn}
 
\begin{defn}[Sequence-Induced Distribution on States]
\label{defn:InducedDistr}
For a PFSA $G = \paren{Q, \Sigma, \delta, \pitilde}$ and a distribution $\wp_0$ on $Q$, the \textbf{distribution on $Q$ induced by a sequence $x$} is given by $\wp^{T}_{G, \wp_0}(x) = \nrm{\wp_0^{T}\Gamma_{x}}$ with $\wp_{G, \wp_0}(\lambda) = \wp_0$. The entry indexed by $q\in Q$ of the vector $\wp_{G, \wp_0}(x)$ is written as $\wp_{G, \wp_0}(x, q)$. When $\wp_0=\wp_{G}$, the stationary distribution of $G$, we write $\wp_{G, \wp_0}(x)$ as $\wp_{G}(x)$, or simply as $\wp(x)$, if $G$ is understood.
\end{defn}

\begin{defn}[Stochastic Process Generated by a PFSA]
\label{defn:StochasticProcessOfPFSA}
Let $G=\paren{Q, \Sigma, \delta, \pitilde}$ be a PFSA and let $\wp_0$ be a distribution on $Q$, the $\Sigma$-valued stochastic process $\set{X_t}_{t\in\Sigma}$ generated by $G$ and $\wp_0$ satisfies that $X_1$ follows the distribution $\wp_0$ and $X_{t+1}$ follows the distribution $\wp_{G, \wp_0}\paren{X_{1}\cdots X_{t}}$ for $t\in\N$.
\end{defn}
For the rest of this paper, we will assume $\wp_0 = \wp_G$ if not specified otherwise. We can show that, when initialized with $\wp_G$, the process generated by a PFSA $G$ is stationary and ergodic. We also note the, for the process generate by $G$, we have $\phi_x = \wp_G(x)^{T}\Pitilde$. Since $\wp_G(\lambda) = \wp_{G}$, the symbolic derivative of the empty sequence $\phi_\lambda$ is the stationary distribution on the symbols. 

\begin{defn}[Synchronizable PFSA and Synchronizing Sequence]
A \textbf{synchronizing sequence} is a finite sequence that sends an arbitrary state of the PFSA to a fixed state \cite{trahtman2008road}. To be more precise, let $G=\paren{Q, \Sigma, \delta, \pitilde}$ be a PFSA, we say a sequence $x\in\Sigma^{\star}$ is a synchronizing sequence to a state $q \in Q$ if $\delta(q', x) = q$ for all $q'\in {Q}$. A PFSA is \textbf{synchronizable} if it has at least one synchronizing sequence. Given a sample path generated by a PFSA, we say the PFSA is \textbf{synchronized} if a synchronizing sequence transpires in the sample path.    
\end{defn}

\begin{defn}[Equivalence and Irreducibility]
Two PFSA $G$ and $H$ are \textbf{equivalent} if they generate the same stochastic process. A PFSA $G$ is said to be \textbf{irreducible}, if there is not another PFSA with smaller state set that is equivalent to $G$. 
\end{defn}
\begin{defn}
Consider a PFSA $G$ over state set $Q$. For a give $\varepsilon > 0$, we say a sequence $x$ is a $\varepsilon$-synchronizing sequence to a state $q\in Q$ if
\cgather{
    \norm{\wp_{G}(x) - \mathbf{e}_q}_{\infty} \leq \varepsilon. 
}
\end{defn}
While there exists PFSA that is not synchronizable, we can show that an irreducible PFSA always has an $\varepsilon$-synchronizing sequence for some state $q$ for arbitrarily small $\varepsilon > 0$. Moreover, we can show that as length increases, sequences produced by PFSA become uniformly $\varepsilon$-synchronizing. These two are the underpinning properties for the inference algorithm of PFSA (See Alg.~\ref{alg:GenESeSS}), because they imply that $\phi_x$ can be used to approximate $\pitilde(q)$ if $x$ are properly prefixed and long enough.

\begin{defn}[Joint $\varepsilon$-Synchronizing Sequence]
\label{def:JointSyncSeq}
Let $G$ and $H$ be two PFSA over state sets $Q_{G}$ and $Q_H$, respectively. For a fixed $\varepsilon$, a sequence $x$ is said to be \textbf{jointly $\varepsilon$-synchronizing} to $(q, r)\in{Q_{G}}\times Q_{H}$ if $x$ is $\varepsilon$-synchronizing to $q$ and to $r$ simultaneously. We define
\cgather{
    \Sigma^{d}_{\varepsilon, (q, r)} \triangleq \set{x\in\Sigma^{d}:\ \textrm{$x$ jointly $\varepsilon$-synchronizing to $(q, r)$}}
}
\end{defn}

\begin{defn}[Joint Pair of States]
Let $G$ and $H$ be two PFSA over state sets $Q_{G}$ and $Q_H$, respectively. Define
\cgather{
    p_{G}(q,r) \triangleq \lim_{d\rightarrow\infty}p_{G}\paren{\Sigma^{d}_{\varepsilon, (q,r)}}
}A pair of states $(q,r)\in Q_G\times Q_H$ is called a \textbf{$G$-joint pair} of states if $p_{G}(q,r) > 0$. We also define
\cgather{
    Q_\textrm{c} \triangleq \set{(q, r)\in Q_G \times Q_H:\ \textrm{$(q, r)$ is a $G$-joint pair}}
}
\end{defn}


The inference algorithm for PFSA is called \textbf{\algo} for \underline{Gen}erator \underline{E}xtraction Using \underline{Se}lf-\underline{s}imilar \underline{S}emantics. With an input sequence $x$ and a hyperparameter $\varepsilon$, \algo outputs a PFSA in the following three steps: 1) approximate an almost synchronizing sequence; 2) identify the transition structure of the PFSA; 3) calculate the transition probabilities of the PFSA. See Alg.~\ref{alg:GenESeSS} for detail.
\begin{algorithm}[!ht]
    \KwData{A sequence $x$ over alphabet $\Sigma$, $0< \varepsilon < 1$}
    \KwResult{State set $Q$, transition map $\delta$, and transition probability $\pitilde$}
    \tcc{\textcolor{PineGreen}{\textbf{Step One: Approximate $\varepsilon$-synchronizing sequence}}}
    Let $L=\ceil{\log_{\abs{\Sigma}}1/\varepsilon}$\; \label{alg:GenL}
    Calculate the \textbf{derivative heap} $\mathcal{D}^{x}_{\varepsilon}$ equaling $\set{\hat{\phi}^{x}_y\ :\ \textrm{$y$ is a sub-sequence of $x$ with } |y|\leq L}$\;
    Let $\mathcal{C}$ be the convex hull of $D^{x}_{\varepsilon}$\; \label{alg:GenConv}
    Select $x_0$ with $\hat{\phi}^{x}_{x_0}$ being a vertex of $\mathcal{C}$ and has the highest frequency in $x$\; \label{alg:GenSyncSeq}
    \tcc{\textcolor{PineGreen}{\textbf{Step Two: Identify transition structure}}}
    Initialize $Q = \set{q_0}$\; \label{alg:GenStep2Start}
    Associate to $q_0$ the \textbf{sequence identifier} $x^{\textrm{id}}_{q_0} = x_0$ and the probability vector $d_{q_0} = \hat{\phi}^x_{x_0}$\;
    Let $\widetilde{Q}$ be the set of states that are just added and initialize it to be $Q$\;
    \While{$\widetilde{Q}\neq\emptyset$}{
        Let $Q_\textrm{new} = \emptyset$ be the set of new states\;
        \For{$(q, \sigma)\in \widetilde{Q} \times \Sigma$}{
            Let $x = x^{\textrm{id}}_q$ and $d = \hat{\phi}^x_{x\sigma}$\;
            \eIf{$\norm{d-d_{q'}}_\infty < \varepsilon$ for some $q'\in Q$\label{alg:GenIdenStateStart}}{
                Let $\delta(q, \sigma) = q'$\;
            }{
                Let $Q_\textrm{new} = Q_\textrm{new}\cup\set{q_\textrm{new}}$ and $Q = Q\cup\set{q_\textrm{new}}$\;
                Associate to $q_\textrm{new}$ the sequence identifier $x_{q_\textrm{new}}^{\textrm{id}} = x\sigma$ and the probability vector $d_{q_\textrm{new}}=d$\;
                Let $\delta(q, \sigma) = q_\textrm{new}$\;
            }\label{alg:GenIdenStateEnd}
        }
        Let $\widetilde{Q} = Q_{\textrm{new}}$\;
    }
    Take a strongly connected subgraph of the labeled directed graph defined by $Q$ and $\delta$, and denote the vertex set of the subgraph again by $Q$\;\label{alg:GenStep2End}
    \tcc{\textcolor{PineGreen}{\textbf{Step Three: Identify transition probability}}}
    Initialize counter $N\bracket{q,\sigma}$ for each pair $(q, \sigma) \in Q\times\Sigma$\; \label{alg:GenIdenTransProbStart}
    Choose a random starting state $q\in Q$\;
    \For{$\sigma\in x$}{
        Let $N\bracket{q,\sigma} = N\bracket{q,\sigma} + 1$\;
        Let $q = \delta\paren{q,\sigma}$\;
    }
    Let $\pitilde\paren{q} = \nrm{\paren{N\bracket{q,\sigma}}_{\sigma\in\Sigma}}$\;\label{alg:GenIdenTransProbEnd}
    \Return $Q$, $\delta$, $\pitilde$\;
    \caption{\algo}
    \label{alg:GenESeSS}
\end{algorithm}

\section{Sequence Likelihood Defect}
\label{sec:SLD}

\begin{defn}[Entropy Rate and KL Divergence]
By entropy rate of a PFSA, we mean the entropy rate of the stochastic process generated by the PFSA\cite{cover2012elements}. Similarly, by KL divergence of two PFSA, we mean the KL divergence between the two processes generated by them~\cite{matthews2016sparse}. More precisely, we have
\cgather{
    \mathcal{H}(G) = -\lim_{d\rightarrow\infty}\frac{1}{d}\sum_{x\in\Sigma^{d}}p(x)\log p(x)
}
and the KL divergence 
\cgather{
    \mathcal{D}_{\textrm{KL}}\parenBar{G}{H} = \lim_{d\rightarrow\infty}\frac{1}{d} \sum_{x\in\Sigma^{d}}p_{G}(x)\log\frac{p_{G}(x)}{p_{H}(x)}
}
whenever the limits exist.
\end{defn}

\begin{thm}[Closed-form Formula for Entropy Rate and KL Divergence]
\label{thm:Closed-formFormulaForEntropyRate}
The entropy rate of a PFSA $G = \paren{\Sigma, Q, \delta, \pitilde}$ is given by 
\cgather{
    \mathcal{H}(G) = \sum_{q\in{Q}}\wp_{G}(q)\cdot h(\pitilde(q))
}
where $h(\mathbf{v})$ is the based-$2$ entropy of the probability vector $\mathbf{v}$.

Consider two PFSA $G = \paren{Q_{\textrm{G}}, \Sigma, \delta_{G}, \pitilde_{G}}$ and $H = \paren{Q_{\textrm{H}}, \Sigma, \delta_{H}, \pitilde_{H}}$ with $\mu_{G}$ being absolutely continuous with respect to $\mu_{H}$. Let $Q_\textrm{c}$ be the set of $G$-joint pairs of states, we have
\cgather{
    \mathcal{D}_{\textrm{KL}}\parenBar{G}{H} = \sum_{(q,r)\in Q_\textrm{c}} p_{G}\paren{q, r}D_{KL}\parenBar{\pitilde_{G}(q)}{\pitilde_{H}(r)}
}
\end{thm}

\begin{defn}[Log-likelihood]
Let $x\in\Sigma^{d}$, the log-likelihood~\cite{cover2012elements} of a PFSA $G$ generating $x$ is given by
\cgather{
    L(x, G) = -\frac{1}{d}\log p_G(x)
}
\end{defn}
The calculation of log-likelihood is detailed in Alg.~\ref{alg:LLK}.
\begin{algorithm}[!ht]
    \KwData{A PFSA $G = \paren{\Sigma, Q, \delta, \pitilde}$ and a sequence $x$ over alphabet $\Sigma$}
    \KwResult{Log-likelihood $L(x, G)$ of $G$ generating $x$}

	Calculate the state transition matrix $\Pi$ and observation $\Pitilde$\; 
	Calculate the stationary distribution over states $\wp_{G}$ of $G$ from $\Pi$\;
	Calculate the stationary distribution of alphabet $\phi_{\lambda}^{T} = \wp_{G}^{T}\Pitilde$\;
	Initialize $\mathbf{p}$ by $\wp_{G}$ and $\mathbf{q}$ by $\phi_{\lambda}$\;
	Let $L = 0$\;
    \For{$i$ from $1$ to $|x|$}{
		Let $\sigma$ be the $i$-th entry of $x$\;
		Let $L = L - \log\mathbf{q}|_\sigma$\;
		Let $\mathbf{p}^{T} = \nrm{\mathbf{p}^{T}\Gamma_{\sigma}}$ where $\Gamma_{\sigma}$ is defined in \ref{defn:GammaExpr}\;
		Let $\mathbf{q}^{T} = \mathbf{p}^{T}\Pitilde$\;
    }
    \Return $L / |x|$;
    \caption{Log-likelihood}
    \label{alg:LLK}
\end{algorithm}

\begin{thm}[Convergence of log-likelihood]
\label{thm:convergenceOfLLH}
Let $G$ and $H$ be two reduced PFSA, and let $x\in\Sigma^d$ be a sequence generated by $G$. Then we have
\cgather{
    L(x, H)\rightarrow \mathcal{H}(G) + \mathcal{D}_{\textrm{KL}}\parenBar{G}{H}
}in probability as $d\rightarrow\infty$.
\end{thm}


\begin{proof}
We first notice that
\calign{
    \sum_{x\in\Sigma^{d}}p_{G}(x)\log\frac{p_{G}(x)}{p_{H}(x)} =& \sum_{x\in\Sigma^{d-1}}\sum_{\sigma\in\Sigma}p_{G}(x)\wp_{G}(x)\left.\Pitilde_{G}\right|_{\sigma}\log\frac{p_{G}(x)\wp_{G}(x)\left.\Pitilde_{G}\right|_{\sigma}}{p_{H}(x)\wp_{H}(x)\left.\Pitilde_{H}\right|_{\sigma}}\\
    =&\sum_{x\in\Sigma^{d-1}}p_{G}(x)\log\frac{p_{G}(x)}{p_{H}(x)} + \underbrace{\sum_{x\in\Sigma^{d-1}}p_{G}(x)\sum_{\sigma\in\Sigma}\wp_{G}(x)\left.\Pitilde_{G}\right|_{\sigma} \log\frac{\wp_{G}(x)\left.\Pitilde_{G}\right|_{\sigma}}{\wp_{H}(x)\left.\Pitilde_{H}\right|_{\sigma}}}_{D_d}
}By induction, we have $\mathcal{D}_{\textrm{KL}}\parenBar{G}{H} = \lim_{d\rightarrow\infty}\frac{1}{d}\sum_{i=1}^{d}D_{i}$, 
and hence by Ces\`{a}ro summation theorem \cite{hardy1992divergent}, we have $\mathcal{D}_{\textrm{KL}}\parenBar{G}{H} = \lim_{d\rightarrow\infty} D_d$. Let $x=\sigma_1\sigma_2...\sigma_n$ be a sequence generated by $G$. Let $x^{[i-1]}$ is the truncation of $x$ at the $(i-1)$-th symbols, we have
\calign{
    -\frac{1}{n}\sum_{i=1}^{n}\log\wp_{H}\paren{x^{[i-1]}}\left.\Pitilde_{H}\right|_{\sigma_i} = \underbrace{\frac{1}{n}\sum_{i=1}^{n}\log\frac{\wp_{G}\paren{x^{[i-1]}}\left.\Pitilde_{G}\right|_{\sigma_{i}}}{\wp_{H}\paren{x^{[i-1]}}\left.\Pitilde_{H}\right|_{\sigma_i}}}_{A_{x, n}} - \underbrace{\frac{1}{n}\sum_{i=1}^{n}\log\wp_{G}\paren{x^{[i-1]}}\left.\Pitilde_{G}\right|_{\sigma_i}}_{B_{x, n}}
}Since the stochastic process $G$ generates is ergodic, we have
\cgather{
    \lim_{n\rightarrow\infty}A_{x,n} = \lim_{d\rightarrow\infty}D_d = \mathcal{D}_{\textrm{KL}}\parenBar{G}{H}
}and $\lim_{n\rightarrow\infty}B_{x,n} = \mathcal{H}(G)$.
\end{proof} 
