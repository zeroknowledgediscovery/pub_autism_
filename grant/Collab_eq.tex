\section*{ACADEMIC PEDIATRICS and
DEVELOPMENTAL AND BEHAVIORAL PEDIATRICS}


University of Chicago Comer Children's Hospital is a 172-bed acute care facility founded in 2005, uniting advanced technology with a family-centered, child-friendly philosophy to provide state-of-the-art care in a six-floor, 242,000-square-feet building. As a major tertiary referral center, Comer Children's sees children with common as well as the most complex medical problems. It admits about 5,000 patients annually from the Chicago area, the Midwest, and around the world. Each year its outpatient clinics accommodate nearly 37,000 general pediatric and specialty visits. Comer Children's is staffed by close to 170 highly trained pediatricians, as well as specially trained nurses and caring support staff, who work together to provide general and specialty medical care for newborns to young adults.

\subsection*{SECTION OF ACADEMIC PEDIATRICS}
The University of Chicago is situated on Chicago’s South Side, a medically underserved area. Our Primary Service area comprises a population of three-quarter of a million people. There is an acute need for primary care practitioners to improve health in the state of Illinois (FY2025 projected shortage -4.0\% and specifically among this population (Chicago South Side Primary Care HPSA score =19). Physicians in this area deliver care that addresses complex medical, psychosocial, and behavioral needs.  They lead efficient, patient-centered primary care practices in underserved communities.

The Section of Academic Pediatrics is dedicated to managing the unique primary care needs of newborns, children and adolescents. The scope of its clinical service covers both the outpatient and inpatient areas located in Comer Children's Hospital and community sites. Its clinical programs include Comer and community hospital medicine, general care nursery, pediatric primary care, Sports Medicine, Child Advocacy and Protective Services and the Comer Mobile Unit.  To support this grant, we will leverage our academic institution and its expanding clinical network across Chicago’s South Side, as well as neighboring federally qualified health centers, to provide diverse and complementary experiences in team-based care. Dr. James Mitchell is the Medical Director for patient services.  

Our provider network includes community health centers and Federally Qualified Health Centers (FQHCs). Sites will include: Comer Children’s Hospital General Pediatrics Clinic, Friend Family Health Center, Erie Family Health Center, ACCESS Community Health Network, and Asian Human Services. All sites have Health Professional Shortage Area designation with scores of 19-21. Some of patient volume statistics are provided as follows:

\textbf{Comer Children's Hospital General Pediatrics Clinic}, located on the University of Chicago Medical campus, serves approximately 12,000 pediatric patients over 22,000 visits annually. Approximately 56% of the patients reside in the South Side community and 35% are insured by Medicaid. This clinic comprises nine pediatricians and one advanced practice nurse.   

\textbf{The Friend Family Health Center} is a large, multi-site Federally Qualified Health Center.   It has served as the primary ambulatory training site for University of Chicago Pediatrics Residency Program since the 1990s.  Friend Health serves over 27,000 patients from Chicago's South Side as part of its mission to provide access to high quality, comprehensive health care   

\textbf{The Erie Family Health Center}, a Federally Qualified Health Center with seven primary care centers and five school-based health centers, provides care to more than 74,000 patients over 300,000 visits annually. These health centers cares for patients across 62 languages. The patient population is 71% Hispanic, with 45% best serviced in a language that is not English. Erie applies team-based care models within their health centers.    

\textbf{ACCESS Community Health Network}, a large Federally Qualified Health Center with 35 health centers across Chicago, cares for more than 183,000 patients per year. ACCESS services the largest proportion of Medicaid beneficiaries in Illinois.  

\textbf{Asian Human Services (AHS)} is a multi-site Federally Qualified Health Center which services a largely immigrant population. AHS provides comprehensive care to more than 30,000 people from more than 55 countries.


\subsection*{THE SECTION OF DEVELOPMENTAL AND BEHAVIORAL PEDIATRICS (DBP)}
The Section of DBP is a Center of Excellence in developmental diagnosis, biomedical management and family supports for children with motor, communicative, sensory, developmental, genetic, neurological, learning and behavior disorders. The section’s goals are to:
\begin{itemize}

\item Promote the highest quality interdisciplinary assessment and biopsychosocial management practices to optimize child functioning, support families and maximize prevention strategies across health, education and community care systems.
\item  Provide the necessary leadership to ensure that children with complex challenges have a high quality medical home and that best practices are used to improve their ability to communicate, move, regulate behavior, interact socially, learn functional and adaptive skills, perform in school and participate in the community.  
\item Serve as a resource for primary care practitioners and pediatric subspecialists for children with the highest biomedical and psychosocial risks associated with suboptimal educational outcomes. • \item Serve as a resource for community professionals and agencies for children with developmental delays, complex disorders after prematurity, genetic, neurological, or cardiopulmonary disorders or experiencing complex behavioral challenges after life-threatening illnesses. . 
\item Enhance training and research so families will benefit from the best clinical and scientific advances with the highest standards of ethics, professionalism and advocacy. 
\item For children with complex medical and/or behavioral challenges and for families with psycho-social or economic stress, DBP provides interventions and assistance. Many of these children and families require teamwork and leadership to promote their health, development, and social competencies. DBP physicians take into account the entire family dynamic and provide guidance in securing ancillary therapies, support services and special school considerations as needed.  
\item We have the clinical expertise coupled with research, educational; advocacy and public policy activities needed for insuring that children receiving translational technologies have information and supports for long term success. Our scope and leadership roles are recognized at the local, regional, national and international levels. We have established clinical, advocacy and research cooperative activities with specialists in community pediatrics, child and adolescent psychiatry, neonatology, critical care medicine, genetics, neurology, child protective services, developmental psychology, public policy and social sciences. These collaborations are not limited to the University of Chicago campus, but extend to other institutions nationally and internationally. Our activities are complementary to those of our collaborators—bringing cross-fertilization of ideas to bear on topics of life course outcomes of children with prematurity. Our Section’s activities are interdisciplinary and include faculty from Public Policy, Social Sciences and Economics to work towards our common interests in improving the quality of life for children with disabilities. 
\end{itemize}

\textbf{The Woodlawn Center} is the clinical home of DBP. It is a site for clinical evaluation and research that can accommodate patients 6 days a week with two exam rooms, equipped to provide developmental assessments from 0-18 years of age. The focus at the Woodlawn Center is on early developmental and cognitive assessments that provide early definition of school-age diagnosis and both clinical and educational needs. 

Interdisciplinary professionals at this center include speech and occupational therapists, social work, developmental psychology, and community outreach professionals.   We provide Autism Diagnostic Observation Schedule (ADOS) testing for accurately assessing and diagnosing autism spectrum disorders.  ADOS is a standardized diagnostic test which provides information on communication, social interaction, and play (or imaginative use of materials) for individuals suspected of having autism or other pervasive developmental disorders, and, as appropriate, provide access to services which may be recommended for each child. .  

DBP and the \textbf{Pediatric Neuropsychology Service} both maintain up to date assessment protocols and manuals for the interdisciplinary assessment of children’s motor, manipulative, conceptual, cognitive, behavioral, and educational skills. These include all the assessment instruments specified in the neurodevelopmental assessment protocol for children in this study. These assessment tools have been used by health professionals representing developmental pediatrics, developmental psychology, physical therapy, occupational therapy, speech language pathology, and special education. A specialized library of current assessment references as well as audiovisual and computer assisted training materials are available.  The Pediatric Neuropsychology Service is a clinical, training and research program within the Section of Child and Adolescent Psychiatry, in the Department of Psychiatry.  The Service provides assessment and Diagnostic services for infants and children with neurodevelopmental learning, emotional regulation and medical disorders and conditions.  The service is also involved in collaborative clinical research, examining the neurocognitive and behavioral sequelae of a wide range of disorders.  

